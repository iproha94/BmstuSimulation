\chapter{Аналитический раздел}

\textbf{Цель} данной работы - реализовать критерий проверки случайности последовательности. Сравнить результаты работы данного критерия на табличных случайных числах и случайных числах, сгенерированных алгоритмически (отдельно для одноразрядных, двухразрядных и трехразрядных). Так же необходимо предусмотреть возможность
задания случайной последовательности вручную.

За эталон генератора случайных чисел (ГСЧ) принят такой генератор, который порождает последовательность случайных чисел с равномерным законом распределения в интервале (0; 1). \cite{stratum-rand} 

Примером физических ГСЧ могут служить: монета («орел» — 1, «решка» — 0); игральные кости; поделенный на секторы с цифрами барабан со стрелкой; аппаратурный генератор шума (ГШ), в качестве которого используют шумящее тепловое устройство, например, транзистор. \cite{stratum-rand} 

В качестве критериев будут использоваться тесты NIST \cite{habrahabr-nist} 