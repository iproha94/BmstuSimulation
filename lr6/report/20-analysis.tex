\chapter{Аналитический раздел}

\textbf{Цель} данной работы - cмоделировать работу касс супермаркета. 

К кассам супермаркета идет поток покупателей. Данный поток - Пуассоновский. Параметр задается.

Каждый покупатель держит несколько товаров. Колличество товаров у пользователя распределено равномерно. 

Есть два типа касс:
\begin{enumerate}
	\item быстрые
	\item обычные
\end{enumerate}

Быстрые кассы обслуживают только тех клиентов, у кого колличество товаров меньше N. N - задается. Количество касс задается.

Обычные кассы обслуживают остальных пользователей. Количество касс задается.

Время обработки кассиром одного товара константное, задается. Время работы кассира равно времени обработки одного товара умноженное на колличество товаров у покупателя.

Вывести колличество пользователей в очереди к каждой кассе в каждый момент времени при заданном колличестве покупателей, которых должен обслужить супермаркет.

\begin{figure}
  \centering
  \includegraphics[scale=0.6]{diag.png}
  \caption{Диаграмма модели}
\end{figure}