\chapter{Аналитический раздел}

\textbf{Цель} данной работы - cмоделировать работу ифнормационного центра. 

В информационный центр приходят клиенты через интервалы времени $10\pm2$ минуты. Если все три имеющихся оператора заняты, то клиенту отказывают в обслуживании. Операторы имеют производительность: $20\pm5$ мин, $40\pm10$, $40\pm20$. Клиенты стараются занять свободного оператора с максимальной производительностью. Полученные запросы сдаются в приемный накопитель, откуда выбираются для обработки на первый компьютер для 1 и 2 оператора и на второй компьютер для 3 оператора. Время обработки первого и второго компьютера 15 и 30 мин.
Смоделировать процесс обработки для 300 запросов. Определить вероятность отказа.
Предусмотреть 300 как на входе, так и на выходе. У компьютеров очереди бесконечные. Оператор освобождается, когда он передаст заявку компьютеру.

\textbf{Результат} При заданных параметрах вероятность отказа клиенту в обслуживании равно 20-25 \%. 